\chapter{Introduction}

\section{Background}

    Helicopters are said to be the only aircraft that, since its conception, has saved more lives than they have taken. Their  high level of mobility, vertical take off and landings and their ability to hover give helicopters great versatility. Helicopters are the most common example of a rotary aircraft and are used in environments ranging from rocky, mountainous to stormy seas. With such high stakes it is vital to minimize points of potential failure. One of these failure points is the helicopter's tail rotor. It is required to counter the toque produced by the engine which rotates the main rotor to produce lift. If the tail rotor were to stop working, the helicopter would lose its controllability and would have to land immediately. A jet-tipped rotary aircraft places the propulsion on the tips of the aircraft's rotor and thus does not produce any toque that needs to be canceled, eliminating the need of a tail rotor. As the tail rotor is connected to the same engine that operates the rotor, transmission of the rotation to the tail rotor increases complexity of the helicopter.
     
    \vspace{3mm}
    This project will research, design, construct and test a jet-tipped rotary aircraft which will actuate the pitch of the rotor  the use of propulsion situated at the tip of the blades. Traditional rotary wing aircraft change the rotor's pitch for portions of its rotation, this creates an unbalanced, causing the aircraft to move in the desired direction. Different methods for directional thrust will investigate varying from traditional methods to using the propulsion force itself to control the direction of the aircraft. 
    \vspace*{3mm}
    
    This project, which is prepared for Mechanical Project 448 and prepared by Mr RA Krüger, was proposed by the student after devising the concept with Dr A Gill. The research and results from this project hopes to further the development of tip-propelled rotary aircraft, which currently is a relatively unresearched field. Stated below include the projects scope, objectives, literature review, motivation and planning of the project are outlined. 

%Starting from the big picture, gradually narrow focus down to this project and where this report fits in. Hello world.

\section{Objectives}

The objectives of the project (in some cases the objectives of the report). If necessary describe limitations to the scope.

\section{Motivation}

Why this specific project/report is worth while.

