\chapter{Objectives}

\section{Scope and limitations}
    As previously mentioned this project aims to research, design, build and test a tip-propelled rotary aircraft. The tip-propulsion should vary the pitch of the rotor such that at lower propulsion, the pitch will have a higher pitch, there by increasing the lift generated. As propulsion is increased, the pitch lessens, decreasing the lift produced, but also decreasing the drag generated by the rotor. The final design should prove controllability, but does not need to achieve sustained flight, and thus showing that the aircraft can produce lift in the desired direction will suffice. While a basic understanding of rotor design can be applied to the aircraft's main rotor, it is not the focus of the project and thus no computational fluid dynamics are required either.

\section{Objectives}
    The objectives of this project are as follows:
    \begin{enumerate}
        \item Construct a working prototype of the created design 
        \item Implement a method to produce directional thrust
        \item Implement a control system for the tip-propulsion
        \item Test and analyse the prototype
    \end{enumerate}
\section{Research questions}
    \subsection{Can the aircraft be fully controlled using the tip propulsion alone?}
    \subsection{What is the efficiency of the aircraft?}
    \subsection{How much thrust can the aircraft produce?}
    \subsection{Which control system method works best?}
    \subsection{How stable is the system?}
    
\section{Motivation}