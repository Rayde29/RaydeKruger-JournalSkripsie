\chapter*{Executive summary}
\vspace*{-1.5cm}
\noindent
\begin{longtable}{|p{\dimexpr \linewidth-2\tabcolsep-2\arrayrulewidth}|}
\hline%------------------------------------------------------------
\sumheading  Title of Project \\
\hline%------------------------------------------------------------
Tip-thurst rotary aircraft \\[1ex]

\hline%------------------------------------------------------------
\sumheading  Objectives \\
\hline%------------------------------------------------------------
 The main objective is to design, build and test a prototype of a tip-thrust rotary aircraft with controllable pitch to create directional thrust. \\[1ex]

\hline%------------------------------------------------------------
\sumheading  What is current practice, and what are its limitations? \\
\hline%------------------------------------------------------------
 Current tip-thrust rotary aircraft either use traditional methods for pitch control, such as a swashplate, or use a fixed pitch rotor with additional methods for propulsion to control direction.  \\[1ex]

\hline%------------------------------------------------------------
\sumheading  What is new in this project? \\
\hline%------------------------------------------------------------
 This project investigates the use of a variation of tip-thrust to control the pitch of the rotor.  \\[1ex]

\hline%------------------------------------------------------------
\sumheading  If the project is successful, how will it make a difference? \\
\hline%------------------------------------------------------------
 The controlling pitch using a variation of thrust will decrease the mechanical complexity of rotary aircraft, removing failure points as well as making then easier to produce, lighter and cheaper. \\[1ex]

\hline%------------------------------------------------------------
\sumheading  What are the risks to the project being a success? Why is it expected to be successful? \\
\hline%------------------------------------------------------------
 There is a risk that the pitch cannot change by a large enough amount or fast enough to allow for directional thrust. This risk is reduced through the correct selection and sizing of the components.\\[1ex]

\hline%------------------------------------------------------------
\sumheading  What contributions have/will other students made/make? \\
\hline%------------------------------------------------------------
 Previous student have studied drones, including the thrust produced by the rotors, which will assist in choosing a propulsion method.\\[1ex]

\hline%------------------------------------------------------------
\sumheading  Which aspects of the project will carry on after completion and why? \\
\hline%------------------------------------------------------------
 The control system and optimization of the design can be made to the proof of concept using the mathematical model. \\[1ex]

\hline%------------------------------------------------------------
\sumheading  What arrangements have been/will be made to expedite continuation? \\
\hline%------------------------------------------------------------
 By documenting the procedure, steps followed, including what each component does and how they interact with each other  as well as a project folder will assist any future further contribution. \\[1ex]

\hline%------------------------------------------------------------
\end{longtable}

