\section*{ECSA Requirements}
\vspace{-5mm} % Adjust this value as needed
\small
% \caption{ECSA Requirements} % Uncomment this line for the table caption
\label{tab:ecsa_requirements}
\begin{longtable}{|c|>{\raggedright\arraybackslash}p{2.4cm}|>{\raggedright\arraybackslash}p{2.2cm}|p{7cm}|} 
\hline
Nr   &    \textbf{ECSA Graduate Attribute (GA)}   &   \textbf{Activity addressing attribute}  &   \textbf{Reasoning} \\ \hline
\endfirsthead % Header for the first page
\hline
Nr   &    \textbf{ECSA Graduate Attribute (GA)}   &   \textbf{Activity addressing attribute}  &   \textbf{Reasoning} \\ \hline
\endhead % Header for subsequent pages
\hline
GA1 & \textbf{Problem-solving}  & \ref{sec: MATLAB_Model}, \ref{sec: design} & The project to create a rotary aircraft with tip-thrust controlled pitch, was broken down systematically using the engineering process. In-depth research into this field was performed, and a mathematical model was created of the system. Using the model, a proof of concept was built that can generate directional thrust. \\ \hline 
GA2 & \textbf{Application of scientific and engineering knowledge} & \ref{sec: MATLAB_Model}, \ref{sec: design}, \ref{sec: Experiment} & To create the mathematical model, an in-depth understanding of fluid mechanics was required. This understanding was used to design the rotors to produce the required lift. This model was then validated by measuring the thrust generated by the proof of concept created.   \\ \hline
GA3 & \textbf{Engineering design} & \ref{sec: design} & The rotors of the aircraft were designed using the mathematical model as a guide. The interfaces between all the components required careful designing as well. To control the aircraft, a circuit was designed to detect pitch and speed, send data wirelessly, control the aircraft, and power the components.   \\ \hline
GA4 & \textbf{Engineering methods, skills and tools, including Information Technology} & \ref{sec: MATLAB_Model}, \ref{sec: Software}, \ref{sec: data_Analysis} & The engineering process guided this project, requiring skills such as technical design, critical thinking, and data analysis to achieve the objectives. Software such as MATLAB was used for data processing as well as creating the Blade Element Analysis program. Software such as Excel was used for managing the budget and the pin-out table for the microcontroller and tracker to validate the sensors. \\ \hline
GA5 & \textbf{Professional and technical communication} & \ref{sec: MATLAB_Model} & Many complex topics were discussed during the creation of the mathematical model. This was done concisely, and each step was explained thoroughly to ensure the reader understands the principles the model is based on. \\ \hline
GA6 & \textbf{Individual, team, and multi-disciplinary working} & \ref{sec: design}, Appendix~\ref{sec: Risk_and_saftey} & Throughout the build process, the student worked with technicians to help build the system as well as organizing to have a lab partner at all times for safety. During the project, the student made many large decisions independently of the supervisor. While the guidance provided was insightful, the student was responsible for the research done, designs made, and proof of concept created. \\ \hline
GA7 & \textbf{Independent learning ability} & \ref{sec: Literature_review}, \ref{sec: MATLAB_Model} & As the student is a Mechatronic engineer, advanced fluid mechanics, such as rotor design, blade element analysis, and actuator disc theory, were vital to understanding the behavior of the system. These concepts had to be studied to ensure a firm understanding of the principle, assumptions, and results before creating the mathematical model.\\ \hline
\end{longtable}
\normalsize

%While creating the mathematical model,creative solutions were required to ensurean accurate model was created. To keepwithin budget, some components